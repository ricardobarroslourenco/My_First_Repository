%% This is file `dib-template.tex',
%% 
%% Copyright 2020 Elsevier Ltd
%% 
%% This file is part of the 'Elsarticle Bundle'.
%% ---------------------------------------------
%% 
%% It may be distributed under the conditions of the LaTeX Project Public
%% License, either version 1.2 of this license or (at your option) any
%% later version.  The latest version of this license is in
%%    http://www.latex-project.org/lppl.txt
%% and version 1.2 or later is part of all distributions of LaTeX
%% version 1999/12/01 or later.
%% 
%% The list of all files belonging to the 'Elsarticle Bundle' is
%% given in the file `manifest.txt'.
%% 
%% Template article for Elsevier's document class `elsarticle'
%% with harvard style bibliographic references
%%
%% $Id: dib-template.tex 185 2020-08-07 09:06:08Z rishi $
%%
%% Use the option review to obtain double line spacing
%\documentclass[times,review,preprint]{elsarticle}

%% Use the options `final' to obtain the final layout
%% Use longtitle option to break abstract to multiple pages if overfull.
%% For Review pdf (With double line spacing)
%\documentclass[times,review]{elsarticle}
%% For abstracts longer than one page.
%\documentclass[times,review,longtitle]{elsarticle}
%% For Review pdf without preprint line
%\documentclass[times,review,nopreprintline]{elsarticle}
%% Final pdf
\documentclass[times,final]{elsarticle.cls}
%%
%\documentclass[times,final,longtitle]{elsarticle}
%%

%%
%% Stylefile to load DIB template
\usepackage{dib}
\usepackage{framed,multirow}

%% The amssymb package provides various useful mathematical symbols
\usepackage{amssymb}
\usepackage{latexsym}

%% For line numbers
%\usepackage[switch]{lineno}

% Following three lines are needed for this document.
% If you are not loading colors or url, then these are
% not required.
\usepackage{url}
\usepackage{xcolor}
\definecolor{newcolor}{rgb}{.8,.349,.1}

%%
\usepackage{longtable}
\usepackage[colorlinks]{hyperref}

\journal{Data in Brief}

\begin{document}

\verso{Barros Lourenco \textit{et al.}}

\begin{frontmatter}

\dochead{Data Article}
%The article title must include the word 'data' or 'dataset'.  
%Please avoid the use of acronyms and abbreviations where possible. 
%For co-submission authors, the title should be unique, 
%i.e. not the same as your research paper. 
%A maximum of 250 characters is allowed.
\title{Soil Organic Carbon estimates derived from remote sensing data\tnoteref{tnote1}}%
\tnotetext[tnote1]{Preliminary work and title - as partial requirement for approval at GEOG 512 - Reproducible Research Workflow.}
%Tip: here are a few examples of recent suitable article titles - these are short and clear:
%%Adolescent Rat Social Play: Amygdalar Proteomic and Transcriptomic Data
%%Execution Data Logs of a Supercomputer Workload Over its Extended Lifetime
%%Calgary Preschool Magnetic Resonance Imaging (MRI) Dataset]

%%Authors
\author[1]{Ricardo \snm{Barros Lourenco}\corref{cor1}}
\cortext[cor1]{Corresponding author.}
\ead{barroslr@mcmaster.ca}
%   Tel.: +0-000-000-0000;  
%   fax: +0-000-000-0000;}
\author[1]{Camile \snm{Sothe}}
% 
\author[1]{Alemu \snm{Gonsamo}}
%% Third author's email

\author[2]{Joyce \snm{Arabian}}
\author[2]{James \snm{Snider}}

%%Affiliations
\address[1]{School of Earth, Environment \& Society, McMaster University, Hamilton, Ontario, Canada}
\address[2]{World Wildlife Fund Canada, Toronto, Ontario, Canada}

%\received{1 May 2013}
%\finalform{10 May 2013}
%\accepted{13 May 2013}
%\availableonline{15 May 2013}
%\communicated{S. Sarkar}


\begin{abstract}
%%%%

This article describes a dataset of Soil Organic Carbon (SOC) estimates for the entire Canadian territory, derived from remote sensing measurements processed by a machine learning model. 

This dataset can be used by scientists involved in climate science studies, more specifically in carbon stock modeling, and carbon cycle monitoring. 

The data is derived from satellite measurements of temperature, precipitation, elevation, terrain slope, vegetation indexes, and radar polarization. These sources were used to train a machine learning model able to predict SOC concentrations on soil. As target for such modelling process, it were used ground measurements of SOC at different depths.

The main article in which the employed methodology and analysis details are provided in full extent is available at Sothe et al (2021)\cite{1}. 
% Please Type your abstract here.

% \noindent[The Abstract should describe the data collection process, the analysis
% performed, the data, and their reuse potential. It should not provide
% conclusions or interpretive insights. If your article is being
% submitted via another Elsevier journal as a co-submission, please cite
% this research article in the abstract.  

% \noindent\textbf{Tip:} do not use words such as
% `study', `results', and `conclusions' because a data article should be
% describing your data only.  Min 100 words - Max 500 words.]
%%%%
\end{abstract}

\begin{keyword}
%% Keywords
%[Include 4-8 keywords (or phrases) to facilitate others finding your
%article online. 
%\noindent\textbf{Tip:} Try Google Scholar to find which terms are most common in your
%field. In biomedical fields, MeSH terms are a good 'common vocabulary'
%to draw from]
\KWD Soil Organic Carbon\sep Remote Sensing\sep Machine Learning
\end{keyword}

\end{frontmatter}

\newpage

% %% For linenumbers
% %\linenumbers
% \section*{Data in Brief Article Template}
% %% main text
% \noindent \href{https://www.journals.elsevier.com/data-in-brief/about-data-in-brief/data-in-brief-faq}%
% {\textit{Data in Brief}} 
% is an open access journal that publishes data articles.
% Please note:
% \begin{itemize}
% \item A data article is different to a research article, so it is
% important to \textbf{use the template} below to prepare your manuscript for Data
% in Brief.
% \item A data article should \textbf{simply describe data} without providing
% conclusions or interpretive insights.
% \item Before you start writing your data article you should read the
% guidance on 
% \href{https://www.journals.elsevier.com/data-in-brief/policies-and-guidelines/what-data-are-suitable-for-data-in-brief}%
% {What Data are Suitable for Data in Brief}.
% \item It is mandatory that Data in Brief authors share their research data:
% \begin{itemize}
% \item If you have \textbf{raw data} (also referred to as primary, source or
% unprocessed data) relating to any charts, graphs or figures in the
% manuscript, these data must be publicly available, either with the data
% article (e.g. as a supplementary file) or hosted on a trusted data
% repository.

% \item If you are describing \textbf{secondary data} you are required to provide
% a list of the primary data sources used \underline{and} to make the full secondary
% dataset publicly available, either with the data article (e.g. as a
% supplementary file) or hosted on a trusted data repository.

% \item Although we allow supplementary files, it is preferred that
% authors deposit their data in a trusted data repository ($>$70\% of
% Data in Brief authors now do this). See our�
% \href{https://www.elsevier.com/authors/author-resources/research-data/data-base-linking#repositories}%
% {list of supported data repositories}.

% \item For data that, for ethical reasons, require access controls a
% mechanism must be provided so that our Editors and reviewers may access
% these data without revealing their identities to authors (more
% information is provided in the template \hyperlink{target1}{below}).
% \end{itemize}
% \end{itemize}

% Have you any questions? See a list of frequently asked questions 
% \href{https://www.journals.elsevier.com/data-in-brief/about-data-in-brief/data-in-brief-faq}{here},
% or email our Managing Editors:  
% \href{mailto:dib-me@elsevier.com}{dib-me@elsevier.com}. This
% step-by-step�
% \href{https://www.journals.elsevier.com/data-in-brief/about-data-in-brief/how-to-submit-your-research-data-article-data-in-brief}%
% {video}�guide will also tell you how to complete the
% template correctly to maximise your chances of acceptance.

% \vskip6pt
% \noindent Authors can submit to Data in Brief in two ways:

% \begin{enumerate}
% \item[\bf(1)] \textbf{If you are submitting your data article directly to Data in
% Brief, you can now skip the next section and complete the} 
% \hyperlink{target2}{\textbf{Data Article template}}.

% \item[\bf(2)] \textbf{If you are submitting your data article to Data in Brief via
% another Elsevier journal as a co-submission (i.e. with a Research
% Article), please read the} \hyperlink{target3}{\textbf{Co-submission Instructions}}
% \textbf{on the next page
% before completing the} \hyperlink{target2}{\textbf{Data Article template}}.

% \end{enumerate}

% \hypertarget{target3}{}
% \section*{Co-submission Instructions}
% A co-submission to~\textit{Data in Brief}~is done at the same time that you
% submit (or resubmit, after revision) a research article to another
% Elsevier journal. For co-submissions you therefore submit your 
% \textit{Data in Brief} data article manuscript via the other journal's 
% submission system and \underline{not} directly to 
% \textit{Data in Brief} itself. 

% \vskip6pt\noindent
% The other Elsevier journal's Guide for Authors will state if a
% co-submission is offered by that journal, and any revision letter/email
% you receive from a participating journal will contain an offer to
% submit a data article to \textit{Data in Brief}.

% \vskip6pt\noindent
% \textbf{To complete a co-submission you will need to zip your
% ~\textit{Data in Brief}~
% manuscript file and all other files relevant to the 
% ~\textit{Data in Brief}~
% submission (including any supplementary data files) into a single .zip
% file, and upload this as a "Data in Brief"-labelled item in the other
% journal's submission system when you submit manuscript to that journal.
% The .zip file will then be automatically transferred to 
% ~\textit{Data in Brief}~
% when your research article is accepted for publication in the other
% journal, and when published your original research article and data
% article will link to each other on ScienceDirect.}

% \vskip6pt\noindent
% \textbf{As~\textit{Data in Brief}~is open access, a moderate article publication charge
% (APC) fee is payable on publication. For more information about the
% APC, please see 
% \href{https://www.elsevier.com/journals/data-in-brief/2352-3409/open-access-journal}%
% {here}.}

% \vskip6pt\noindent
% \textit{Data in Brief} \underline{requires} that authors share their research data. This can
% be done by submitting it with the data article (e.g. as a supplementary
% file) or by hosting on a trusted data repository (the latter is
% preferred). Failure to do this will delay publication of your
% co-submission.

% \vskip6pt\noindent
% \textbf{If you have any questions, please contact: 
% \href{mailto:DIB@Elsevier.com}{DIB@Elsevier.com}}

% \vskip6pt\noindent
% Please note, authors should not republish the same data presented in
% their original research article in a \textit{Data in Brief} co-submission, as
% this could constitute duplicate publication; however, \textit{Data in Brief}
% welcomes the publication of any data article that fulfils one or more
% of the following criteria:

% \checkmark A description of the supplementary data that would
% previously have been hosted as supplementary electronic files alongside
% your original research article.*

% \checkmark A description of the full dataset or additional information
% that will aid reuse of the data.

% \checkmark A detailed description of the raw data relating to the
% charts, graphs or figures in your companion research article, if making
% these data available will substantially enhance reproducibility and/or
% reanalysis of the data.

% \checkmark Any negative datasets or data from intermediate experiments
% related to your research. 

% \textsf{X} Review articles or supplemental files from a review article
% are not considered original data and are typically unsuitable for Data
% in Brief. 

% \vskip12pt\noindent
% * If describing supplementary data that you previously planned to
% publish as supplem
% entary electronic files hosted alongside the original
% research article, it is requested that you either\break 
% deposit these in a
% repository (preferred) or submit these to \textit{Data in Brief} alongside the
% data article. \textbf{They should not be published as supplementary files with
% your research article in the other journal}.

% \clearpage
% \hypertarget{target2}{}
% \section*{Data Article template}
% \noindent
% Please fill in the template below. All sections are mandatory unless
% otherwise indicated. Please read all instructions in [square brackets]
% carefully and ensure that you delete all instruction text (including
% the questions) from the template before submitting your article. 

% \vskip6pt\noindent
% Reminder: A data article simply describes data and should not provide
% conclusions or interpretive insights, so \textbf{avoid} using words such as
% `study', `results' and `conclusions'.  

% \vskip6pt\noindent
% We would welcome feedback on this template and how it might be
% improved. To provide anonymous feedback via a very short survey, please
% click \href{https://forms.office.com/Pages/ResponsePage.aspx?id=P-50kiWUCUGif5-xXBBnXTeXkbO343VFrbpYVBvxdZtUM05UVjIwM0U4WlRKUldCOTNMRUQwOVRHTy4u}%
% {here}. 

% \vskip6pt\noindent
% {\small\textbf{\textit{Please delete this line and everything above it before submitting your
% article, in addition to anything in [square brackets] below, including
% in the Specifications Table}}\vskip6pt\hrule\vskip12pt}

{\fontsize{7.5pt}{9pt}\selectfont
%%%
\noindent\textbf{Specifications Table} 

Every section of this table is mandatory. 
Please enter information in the right-hand column and remove all the instructions
\begin{longtable}{|p{33mm}|p{94mm}|}
\hline
\endhead
\hline
\endfoot
Subject                & Computers in Earth Sciences\\
\hline                         
Specific subject area  & Machine-learning generated data (precisely Soil Organic Carbon Estimates), using remote sensing data as covariates.\\
\hline
Type of data           & Multi-channel Georeferenced Image \newline \\             
%\clearpage
\newline                  
How data were acquired & Data generated by a machine learning model (Quantile                                Regression Random Forests) on a covariate set derived from                           satellite imagery bands. The target variable are carbon                              estimates obtained from field surveys across Canada. \\ 
                        % [State how the data were acquired: E.g. Microscope,  
                        %  SEM, NMR, mass spectrometry, survey* etc.\newline
                        %  Instruments: E.g. hardware, software, program\newline
                        %  Make and model and of the instruments used:\newline

                        %  {\fontsize{7pt}{8pt}\selectfont
                        %  *\,if you conducted a survey you must submit a copy of the 
                        %  survey(s) used (either provide these as supplementary material 
                        %  file or provide a URL link to the survey 
                        %  in this section of the table). 
                        %  If the survey is not written in English, 
                        %  please provide an English-language translation.}]
                        %  \\
\hline                         
Data format            & Georeferenced multi-channel raster files stored in WebService.\\
\hline                         
Parameters for         
data\newline 
collection             & The remote sensing data was collected for periods between                          5-20 years based on availability (for cloud removal), with a                         temporal upsampling starting on 5 days (of recoverage). The                           scale of analysis also was set in 250m (with downsampling                           starting in 1km and upsampling starting on 30m). \\  

\hline
Description of          
data\newline 
collection             & Satellite data is orbital sun-synchronous satellite data,                          using multispectral sensors. SOC data was acquired on field                          campaigns over the years. \\
\hline                         
Data source location   & Primary data sources:  \newline
                        \newline
                        USGS Landsat 8 Surface Reflectance Tier 1
                        Red band 0.64–0.67 µm, NIR band 0.85–0.88 µm,
                        SWIR1 band 1.57–1.65 µm, SWIR2 band 2.11–2.29 µm \newline 
                        \newline
                        Landsat 8 Collection 1 Tier 1 32-Day NDWI Composite - Normalized Difference Water Index (annual)
                        \newline \newline
                        Landsat 8 Collection 1 Tier 1 8-Day NDSI Composite - Normalized Difference Snow Index (annual)
                        \newline \newline
                        Tasseled cap transformation based on Landsat 8 at-satellite reflectance (annual - brightness/greenness/wetness)
                        \newline \newline
                    	USGS Landsat 8 Surface Reflectance Tier 1
                        Brightness temperature band 10.60–11.19 µm (bimonthly except Nov to Feb) / 
                       Brightness temperature band 11.50–12.51 µm (bimonthly except Nov to Feb)
                        \newline\newline
                        MODIS Terra Vegetation Indices 16-Day Global 250 m (NDVI) - Normalized Difference Vegetation Index (bimonthly)
                        \newline \newline
                    	MODIS Terra Vegetation Indices 16-Day Global 250 m (EVI) - Enhanced Vegetation Index (annual)
                    	\newline \newline
                    	MODIS Global Terrestrial Evapotranspiration 8-Day Global 1 km
                    	\newline \newline
                    	MODIS Terra Land Surface Temperature and Emissivity Daily (Day and Night) Global 1 km
                    	\newline \newline
                    	MODIS Long-term Land Surface Temperature daytime monthly standard deviation
                    	\newline\newline
                    	Daymet - Daily surface maximum 2-meter air temperature / Daily surface minimum 2-meter air temperature / Daily incident shortwave radiation flux density / Daily total precipitation, sum of all forms converted to water-equivalent
                    	\newline \newline
                    	Digital Elevation Model (ALOS) and slope derived from it
                    	\newline \newline
                    	Global PALSAR-2/PALSAR Yearly Mosaic, converted to decibels (DB) / 
                        L-band duo-polarization horizontal transmit/horizontal receive (HH) and horizontal transmit/vertical receive (HV)
                        \newline \newline
                        Calculated topographic position index \cite{2}
                        \newline \newline
                        Calculated terrain ruggedness index \cite{3}
                        \newline \newline
                        Soil types and depths of Soil Landscape of Canada (SLC) 
                        \\

% [Fill in the information available, and delete from this list as appropriate:\newline

%                          Institution:\newline
%                          City/Town/Region:\newline
%                          Country:\newline
%                          Latitude and longitude (and GPS coordinates, if possible) for collected samples/data:\newline


%                          If you are describing secondary data, you are required to provide a list of 
%                          the primary data sources used in the section.\newline

                        %  Primary data sources:  ]\\
\hline                         
\hypertarget{target1}
{Data accessibility}   & The dataset will be stored on the Google Earth Engine 
                        Platform\cite{5}, using either a Google Drive mount, or a Google Cloud Storage.
                        \newline \newline
                        In case of impossibility, it will be stored on premises of McMaster University, and Indexed with Globus \cite{6}.\\

% [State here if the data are either hosted `With the article' or on a public repository. 
%                          In the interests of openly sharing data we recommend hosting your data in a 
%                          trusted repository ($>$70\% of Data in Brief authors now use a data repository). 
%                          See our \href{https://www.elsevier.com/authors/author-resources/research-data/data-base-linking#repositories}{list of supported data repositories}. 
%                          We suggest \href{https://data.mendeley.com/}{Mendeley Data} if you do not have a trusted repository.\newline

%                          Please delete or complete as appropriate, either:]\newline

%                          With the article\newline

%                          [Or, if in a public repository:]\newline

%                          Repository name: [Name repository]\newline
%                          Data identification number: [provide number]\newline
%                          Direct URL to data: [e.g. https://www.data.edu.com - please note,\newline 
%                          this URL should be working at the time of submission]\newline

%                          [\textbf{In addition, for data with access controls only:} For data that, 
%                          for ethical reasons (i.e. human patient data), 
%                          require access controls please describe
%                          how readers can request access these data and provide a link to any 
%                          Data Use Agreement (DUA) or upload a copy as a supplementary file.]\newline

%                          Instructions for accessing these data:\newline

%                          [\textit{Important: if your data have access controls a mechanism must also be 
%                          provided so that our Editors and reviewers may access these data 
%                          without revealing their identities to authors, please include 
%                          these instructions with your submission. Please contact the Managing 
%                          Editors (DIB-ME@DIB.com) if you have any questions.}]\\                         
\hline                         
Related                 
research\newline
article                &  This article is directly related with:\newline\newline
                            Camile Sothe, Alemu Gonsamo, Joyce Arabian, James Snider, Large scale mapping of soil organic carbon concentration with 3D machine learning and satellite observations, Geoderma, Volume 405, 2022, 115402, ISSN 0016-7061, https://doi.org/10.1016/j.geoderma.2021.115402.

% [If your data article is related to a research article - \textbf{especially 
%                          if it is a co-submission} - please cite your associated research 
%                          article here. Authors should only list \textbf{one article}.\newline

%                          Authors' names\newline
%                          Title\newline
%                          Journal\newline
%                          DOI: \textbf{OR} for co-submission manuscripts `In Press'\newline

%                          \textbf{For example, for a direct submission:}\newline

%                          J. van der Geer, J.A.J. Hanraads, R.A. Lupton, The art of writing a scientific article, 
%                          J. Sci. Commun. 163 (2010) 51-59. https://doi.org/10.1016/j.Sc.2010.00372\newline

%                          \textbf{Or, for a co-submission (when your related research article has not yet published):}\newline

%                          J. van der Geer, J.A.J. Hanraads, R.A. Lupton, The art of writing a 
%                          scientific article, J. Sci. Commun. In Press.\newline

%                          \textbf{Or, if your data article is not directly related to a research article, 
%                          please delete this last row of the table.}]
\end{longtable}
}
%%%            

\section*{Value of the Data}

[Provide 3-6 bullet points explaining why these data are of value to the scientific community. 
Bullet points 1-3 must specifically answer the questions next to the bullet point, 
but do not include the question itself in your answer. You may 
provide up to three additional bullet points to outline the value of these data. 
Please keep points brief, with ideally no more than 400 characters for each point.]

\begin{itemize}
\itemsep=0pt
\parsep=0pt
\item This dataset is important, because it is the first attempt to systematically estimate Canadian soil carbon stocks with usage of machine learning and remote sensing.
\item Potential beneficiaries are climate change policy makers, and stakeholders involved in carbon cycle studies.
\item This data may be used as input in other models, given that the uncertainties need to be propagated across models (since it is a model run output).
\item Broader impacts of this dataset may lie on the ability for policymakers and the general public to assess current carbon stock location, and due to that evaluate proposed policies, and risks/benefits derived from actions of the government to mitigate climate change. 
% \item Your first bullet point must explain why these data are useful or important? 
% \item Your second bullet point must explain who can benefit from these data?
% \item Your third point bullet must explain how these data might be used/reused for 
% further insights and/or development of experiments.
% \item In the next three points you may like to explain how these data could 
% potentially make an impact on society and highlight any other additional value of these data.
% \item ....
\end{itemize}

\section*{Data Description}
[\textit{This is still preliminary work. Once I reprocess the data, I intend to add up to this section.}]
% \noindent [Individually describe each data file (i.e. figure 1, figure 2, table
% 1, dataset, raw data, supplementary data, etc.) that are included in
% this article. Please make sure you refer to every data file and provide
% a clear description for each - do not simply list them. No insight,
% interpretation, background or conclusions should be included in this
% section. Please include legends with any tables, figures or graphs.

% \noindent\textbf{Tip:} do not forget to describe any supplementary data files.]

\section*{Experimental Design, Materials and Methods}

[\textit{This is still preliminary work. Once I reprocess the data, I intend to add up to this section.}]
% \noindent [Offer a complete description of the experimental design and methods
% used to acquire these data. Please provide any programs or code files
% used for filtering and analyzing these data. It is very important that
% this section is as comprehensive as possible. If you are submitting via
% another Elsevier journal (a co-submission) you are encouraged to
% provide more detail than in your accompanying research article. There
% is no character limit for this section; however, no insight,
% interpretation, or background should be included in this section.

% \noindent\textbf{Tip:} do not describe your data (figures, tables, etc.) in this section,
% do this in the Data Description section above.]  

\section*{Ethics Statement}
This work does not involve either human subjects, or animal experiments.
% \noindent [Please refer to the journal's 
% \href{https://www.elsevier.com/journals/data-in-brief/2352-3409/guide-for-authors}{Guide for Authors} 
% for more information on
% the ethical requirements for publication in Data in Brief. In addition
% to these requirements:

% \noindent\textbf{If the work involved the use of human subjects:}
% please include a statement here confirming that informed consent was
% obtained for experimentation with human subjects; 

% \noindent\textbf{If the work involved animal experiments:} please
% include a statement confirming that all experiments comply with
% the \href{https://www.nc3rs.org.uk/arrive-guidelines}{ARRIVE\ guidelines} and were be carried out in accordance with the
% U.K. Animals (Scientific Procedures) Act, 1986 and associated
% guidelines, \href{https://ec.europa.eu/environment/chemicals/lab_animals/legislation_en.htm}{EU Directive 2010/63/EU for animal experiments}, or the
% National Institutes of Health guide for the care and use of Laboratory
% animals (NIH Publications No. 8023, revised 1978)]

\section*{Acknowledgments}
This work was supported by World Wildlife Fund Canada (WWF-Canada).

A.G. acknowledges funding from Natural Sciences and Engineering Research Council of Canada Discovery Grant (RGPIN-2020-05708) and the Canada Research Chairs Program.

R. B. L. acknowledges the careful review provided by professors and colleagues at GEOG 512 - Reproducible Data Workflow.

\section*{Declaration of Competing Interest}

The authors declare that they have no known competing financial interests or personal relationships that could have appeared to influence the work reported in this paper.

% \section*{References}

% \noindent [References are limited (approx. 15) and excessive self-citation is not
% allowed. \textbf{If your data article is co-submitted via another Elsevier
% journal, please cite your associated research article here.}

% \noindent\textbf{Reference style:}
% Text:�Indicate references by number(s) in square brackets in line with
% the text. The actual authors can be referred to, but the reference
% number(s) must always be given.�

% \noindent Example: '..... as demonstrated [3,6]. Barnaby and Jones [8] obtained a different result ....'�

% \noindent [Use \verb+\cite+ command to cite a reference list item in text.

% \noindent These are examples for reference citations \cite{1}.
% \cite{2}. 
% \cite{4}.]

% \subsection*{Reference list using Bib\TeX database file}
% \noindent [If Bib\TeX database file is used for reference data please use 
% \begin{verbatim}
% \bibliographystyle{model1-num-names}
% \bibliography{<BibTeX file name>}
% \end{verbatim}
% and use bibtex command to generate list of references.]

% %% Numbered
% %%If 
% \bibliographystyle{model1-num-names}
% \bibliography{refs}

% \subsection*{Reference list using bibliography envrironment}

% \noindent [List:�Number the references (numbers in square brackets) in
% the list in the order in which they appear in the text.�

% \noindent Examples:�

% {\small
% \begin{verbatim}
% \begin{thebibliography}{0}
% \bibitem{1} J. van der Geer, J.A.J. Hanraads, R.A. Lupton, The art of
% writing a scientific article, J. Sci. Commun. 163 (2010) 51-59.
% https://doi.org/10.1016/j.Sc.2010.00372.
% \bibitem{2} Van der Geer, J., Hanraads, J.A.J., Lupton, R.A., 2018. The
% art of writing a scientific article. Heliyon. 19, e00205.
% https://doi.org/10.1016/j.heliyon.2018.e00205.
% \bibitem{3} W. Strunk Jr., E.B. White, The Elements of Style, fourth ed., 
% Longman, New York, 2000.�
% \bibitem{4} G.R. Mettam, L.B. Adams, How to prepare an electronic
% version of your article, in: B.S. Jones, R.Z. Smith (Eds.),
% Introduction to the Electronic Age, E-Publishing Inc., New York, 2009,
% pp. 281-304.
% \bibitem{5} Cancer Research UK, Cancer statistics reports for the UK.
% http://www.cancerresearchuk.org/aboutcancer/statistics/cancerstatsreport/, 
% 2003 (accessed 13 March 2003).
% \bibitem{6} [dataset] M. Oguro, S. Imahiro, S. Saito, T. Nakashizuka,
% Mortality data for Japanese oak wilt disease and surrounding forest
% compositions, Mendeley Data, v1, 2015.
% https://doi.org/10.17632/xwj98nb39r.1.
% \end{thebibliography}
% \end{verbatim}
% }

% \noindent Example:]
% \vspace*{-12pt}

\begin{thebibliography}{0}
\bibitem{1} Camile Sothe, Alemu Gonsamo, Joyce Arabian, James Snider,
Large scale mapping of soil organic carbon concentration with 3D machine learning and satellite observations, Geoderma, Volume 405, 2022, 115402, ISSN 0016-7061,
https://doi.org/10.1016/j.geoderma.2021.115402.

\bibitem{2} Wilson, J.P., Gallant, J.C., 2000. Terrain Analysis - Principles and Applications. 512p., ISBN: 978-0-471-32188-0.

\bibitem{3} Riley, Shawn J., Stephen D. DeGloria, and Robert Elliot. "Index that quantifies topographic heterogeneity." intermountain Journal of sciences 5.1-4 (1999): 23-27.

\bibitem{4} National Soil Database. Soil Landscape of Canada version 3.2. (2011) \newline
http://sis.agr.gc.ca/cansis/nsdb/slc/index.html

\bibitem{5} Gorelick, N., Hancher, M., Dixon, M., Ilyushchenko, S., Thau, D., & Moore, R. (2017). Google Earth Engine: Planetary-scale geospatial analysis for everyone. Remote sensing of Environment, 202, 18-27.

\bibitem{6} Rachana Ananthakrishnan, Ben Blaiszik, Kyle Chard, Ryan Chard, Brendan McCollam, Jim Pruyne, Stephen Rosen, Steven Tuecke, and Ian Foster. 2018. Globus Platform Services for Data Publication. In Proceedings of the Practice and Experience on Advanced Research Computing (PEARC '18). Association for Computing Machinery, New York, NY, USA, Article 14, 1–7. DOI:https://doi.org/10.1145/3219104.3219127
% \item[] \textbf{Reference to a journal publication:}
% \bibitem{1} J. van der Geer, J.A.J. Hanraads, R.A. Lupton, The art of
% writing a scientific article, J. Sci. Commun. 163 (2010) 51-59.
% https://doi.org/10.1016/j.Sc.2010.00372.

% \textbf{Reference to a journal publication with an article number:}
% \bibitem{2} Van der Geer, J., Hanraads, J.A.J., Lupton, R.A., 2018. The
% art of writing a scientific article. Heliyon. 19, e00205.
% https://doi.org/10.1016/j.heliyon.2018.e00205.

% \textbf{Reference to a book:}
% \bibitem{3} W. Strunk Jr., E.B. White, The Elements of Style, fourth ed., Longman, New York, 2000.�

% \textbf{Reference to a chapter in an edited book:}
% \bibitem{4} G.R. Mettam, L.B. Adams, How to prepare an electronic
% version of your article, in: B.S. Jones, R.Z. Smith (Eds.),
% Introduction to the Electronic Age, E-Publishing Inc., New York, 2009,
% pp. 281-304.

% \textbf{Reference to a website:}
% \bibitem{5} Cancer Research UK, Cancer statistics reports for the UK.\newline
% http://www.cancerresearchuk.org/aboutcancer/statistics/cancerstatsreport/, 2003 (accessed 13 March 2003).

% \textbf{Reference to a dataset:}
% \bibitem{6} [dataset] M. Oguro, S. Imahiro, S. Saito, T. Nakashizuka,
% Mortality data for Japanese oak wilt disease and surrounding forest
% compositions, Mendeley Data, v1, 2015.
% https://doi.org/10.17632/xwj98nb39r.1.] 
\end{thebibliography}

\end{document}

%%